\documentclass[english]{article}
\usepackage{fullpage}
\usepackage{setspace}
\usepackage{color}
\usepackage{hyperref}
\usepackage{amssymb}
\usepackage{amsmath}
\usepackage{mathtools}
\usepackage[T1]{fontenc}
\usepackage[latin9]{inputenc}
\usepackage{amsthm}
\usepackage{amsmath}
\usepackage{amssymb}
\usepackage{graphicx}
\usepackage{dsfont}
\usepackage{cite}
\usepackage{amsmath}
\usepackage{array}
\usepackage{multirow}
\usepackage{caption}
\usepackage{color}


\usepackage{multicol}

\theoremstyle{plain}
\newtheorem{claim}{\protect\claimname}
\theoremstyle{plain}
\newtheorem{prop}{\protect\propositionname}
\theoremstyle{plain}
\newtheorem{lem}{\protect\lemmaname}
\theoremstyle{plain}
\newtheorem{thm}{\protect\theoremname}
\theoremstyle{plain}
\newtheorem{cor}{\protect\corollaryname}  
\theoremstyle{definition}
\newtheorem{defn}{\protect\definitionname}
\theoremstyle{definition}
\newtheorem{assump}{\protect\assumptionname}
\theoremstyle{definition}
\newtheorem{rem}{\protect\remarkname}

\makeatother
  
\usepackage{babel} 

\providecommand{\claimname}{Claim}
\providecommand{\lemmaname}{Lemma}
\providecommand{\propositionname}{Proposition}
\providecommand{\theoremname}{Theorem}
\providecommand{\corollaryname}{Corollary} 
\providecommand{\definitionname}{Definition}
\providecommand{\assumptionname}{Assumption}
\providecommand{\remarkname}{Remark}

\newcommand{\overbar}[1]{\mkern 1.25mu\overline{\mkern-1.25mu#1\mkern-0.25mu}\mkern 0.25mu}
\newcommand{\overbarr}[1]{\mkern 3.5mu\overline{\mkern-3.5mu#1\mkern-0.75mu}\mkern 0.75mu}
\newcommand{\mytilde}[1]{\mkern 0.5mu\widetilde{\mkern-0.5mu#1\mkern-0mu}\mkern 0mu}
\newcommand{\mytildee}[1]{\mkern 1.5mu\widetilde{\mkern-1.5mu#1\mkern-0mu}\mkern 0mu}
  
\DeclareMathOperator*{\argmax}{arg\,max}
\DeclareMathOperator*{\argmin}{arg\,min}
\DeclareMathOperator*{\maximize}{maximize}
\DeclareMathOperator*{\minimize}{minimize} 
\newcommand{\openone}{\mathds{1}}

\newcommand{\todo}{\bf \color{red} [TODO]~}

% Lecture notes
\newcommand{\Loss}{\mathrm{Loss}}
\newcommand{\wv}{\mathbf{w}}
\newcommand{\ptilde}{\widetilde{p}}
\newcommand{\bzeta}{\boldsymbol{\zeta}}
\newcommand{\bPhi}{\boldsymbol{\Phi}}
\newcommand{\Dtrain}{\mathcal{D}_{\mathrm{train}}}
\newcommand{\Dval}{\mathcal{D}_{\mathrm{val}}}
\newcommand{\Dtest}{\mathcal{D}_{\mathrm{test}}}
\newcommand{\Mc}{\mathcal{M}}
\newcommand{\mineq}{m_{\mathrm{ineq}}}
\newcommand{\meq}{m_{\mathrm{eq}}}
\newcommand{\ferm}{f_{\mathrm{erm}}}
\newcommand{\err}{\mathrm{err}}
\newcommand{\dVC}{d_{\mathrm{VC}}}
\newcommand{\Fall}{\mathcal{F}_{\mathrm{all}}}
\newcommand{\fopt}{f_{\mathrm{opt}}}
\newcommand{\BIC}{\mathrm{BIC}}
\newcommand{\AIC}{\mathrm{AIC}}
\newcommand{\bdelta}{\boldsymbol{\delta}}
\newcommand{\Rchat}{\widehat{\mathcal{R}}}

% Ranking and bandits
\newcommand{\thetamin}{\theta_{\mathrm{min}}}
\newcommand{\thetamax}{\theta_{\mathrm{max}}}
\newcommand{\Xhat}{\widehat{X}}
\newcommand{\Ghat}{\widehat{G}}
\newcommand{\jworst}{j_{\mathrm{worst}}}
\newcommand{\Actil}{\widetilde{\mathcal{A}}}
% \newcommand{\Stil}{\widetilde{S}}
\newcommand{\mab}{\texttt{mab}}
\newcommand{\cmp}{\texttt{cmp}}
\newcommand{\pemab}{P_{\mathrm{e},\texttt{mab}}}
\newcommand{\pecmp}{P_{\mathrm{e},\texttt{cmp}}}
\newcommand{\btheta}{\boldsymbol{\theta}}
\newcommand{\Alg}{\mathrm{ALG}}
\newcommand{\Geometric}{\mathrm{Geometric}}
\newcommand{\Ttotal}{T_{\mathrm{total}}}

% NoisyDD
\newcommand{\NDhat}{\widehat{\mathrm{ND}}}
\newcommand{\PDhat}{\widehat{\mathrm{PD}}}
% \newcommand{\Dhat}{\widehat{\mathrm{D}}}
\newcommand{\Shat}{\widehat{S}}
\newcommand{\peid}{P_{\mathrm{e},1}^{(\mathrm{D})}}
\newcommand{\peind}{P_{\mathrm{e},1}^{(\mathrm{ND})}}
\newcommand{\peiid}{P_{\mathrm{e},1}^{(\mathrm{D})}}
\newcommand{\peiind}{P_{\mathrm{e},1}^{(\mathrm{ND})}}
\newcommand{\epsid}{\epsilon_{1}^{(\mathrm{D})}}
\newcommand{\epsind}{\epsilon_{1}^{(\mathrm{ND})}}
\newcommand{\epsiid}{\epsilon_{2}^{(\mathrm{D})}}
\newcommand{\epsiind}{\epsilon_{2}^{(\mathrm{ND})}}
\newcommand{\nid}{n_1^{(\mathrm{D})}}
\newcommand{\nind}{n_1^{(\mathrm{ND})}}
\newcommand{\niid}{n_2^{(\mathrm{D})}}
\newcommand{\niind}{n_2^{(\mathrm{ND})}}
\newcommand{\Npd}{N_{\mathrm{pos}}^{(\mathrm{D})}}
\newcommand{\Npnd}{N_{\mathrm{pos}}^{(\mathrm{ND})}}
\newcommand{\npd}{n_{\mathrm{pos}}^{(\mathrm{D})}}
\newcommand{\npnd}{n_{\mathrm{pos}}^{(\mathrm{ND})}}
\newcommand{\npos}{n_{\mathrm{pos}}}
\newcommand{\nneg}{n_{\mathrm{neg}}}
\newcommand{\Ntilpos}{\widetilde{N}_{\mathrm{pos}}}
% \newcommand{\Npos}{N_{\mathrm{pos}}}
% \newcommand{\Nneg}{N_{\mathrm{neg}}}
\newcommand{\Aid}{\mathcal{A}_{1,i}^{(\mathrm{D})}}
\newcommand{\Aind}{\mathcal{A}_{1,i}^{(\mathrm{ND})}}
\newcommand{\mutild}{\tilde{\mu}^{(\mathrm{D})}}
\newcommand{\mutilnd}{\tilde{\mu}^{(\mathrm{ND})}}

% OneDimBO

\newcommand{\diag}{\mathrm{diag}}
\newcommand{\cunder}{\underline{c}}
\newcommand{\cbar}{\overline{c}}
\newcommand{\mutil}{\widetilde{\mu}}
\newcommand{\sigmatil}{\widetilde{\sigma}}
\newcommand{\Tearly}{T_{\mathrm{early}}}
\newcommand{\Tlate}{T_{\mathrm{late}}}
\newcommand{\Rearly}{R_{\mathrm{early}}}
\newcommand{\Rlate}{R_{\mathrm{late}}}
\newcommand{\Ctil}{\widetilde{C}}
\newcommand{\ctil}{\widetilde{c}}
\newcommand{\ftil}{\widetilde{f}}
\newcommand{\UCB}{\mathrm{UCB}}
\newcommand{\LCB}{\mathrm{LCB}}
\newcommand{\UCBtil}{\widetilde{\mathrm{UCB}}}
\newcommand{\LCBtil}{\widetilde{\mathrm{LCB}}}
\newcommand{\Lctil}{\widetilde{\mathcal{L}}}

% NoisyAdaptiveGT

\newcommand{\nMIi}{n_{\mathrm{MI,1}}}
\newcommand{\nMIii}{n_{\mathrm{MI,2}}}
\newcommand{\nConc}{n_{\mathrm{Conc}}}
\newcommand{\nIndiv}{n_{\mathrm{Indiv}}}
\newcommand{\nHalving}{n_{\mathrm{Halving}}}
\newcommand{\kmax}{k_{\mathrm{max}}}
\newcommand{\ptil}{\widetilde{p}}
\newcommand{\elltil}{\widetilde{\ell}}
\newcommand{\Stil}{\widetilde{S}}
\newcommand{\ncheck}{\check{n}}

% SeparateGT

\newcommand{\cmean}{c_{\mathrm{mean}}}
\newcommand{\cvar}{c_{\mathrm{var}}}
\newcommand{\ccov}{c_{\mathrm{cov}}}
\newcommand{\cmax}{c_{\mathrm{max}}}

\newcommand{\dneg}{d_{\mathrm{neg}}}
\newcommand{\dpos}{d_{\mathrm{pos}}}
\newcommand{\Nneg}{N_{\mathrm{neg}}}
\newcommand{\Npos}{N_{\mathrm{pos}}}
\newcommand{\Nerr}{N_{\mathrm{err}}}
\newcommand{\alphaneg}{\alpha_{\mathrm{neg}}}
\newcommand{\alphapos}{\alpha_{\mathrm{pos}}}

\newcommand{\pej}{P_{\mathrm{e},j}}
\newcommand{\nusymm}{\nu_{\mathrm{symm}}}

% PooledData
\newcommand{\bpi}{\boldsymbol{\pi}}
\newcommand{\Nrm}{\mathrm{N}}
\newcommand{\CNrm}{\mathrm{CN}}

% ConverseGP
\newcommand{\vbar}{\overline{v}}

% GT_ICASSP

\newcommand{\pc}{P_{\mathrm{c}}}

% ActiveMRF

\newcommand{\lammin}{\lambda_{\mathrm{min}}}
\newcommand{\lammax}{\lambda_{\mathrm{max}}}
\newcommand{\missing}{\ast}
\newcommand{\pFAi}{p_{\mathrm{FA},i}}
\newcommand{\pMDi}{p_{\mathrm{MD},i}}
\newcommand{\bSigmatil}{\widetilde{\mathbf{\Sigma}}}
\newcommand{\mtil}{\widetilde{m}}
\newcommand{\Ptil}{\widetilde{P}}
\newcommand{\dmin}{d_{\mathrm{min}}}
\newcommand{\davg}{d_{\mathrm{avg}}}
\newcommand{\dmaxbar}{\overline{d}_{\mathrm{max}}}
\newcommand{\taumin}{\tau_{\mathrm{min}}}
\newcommand{\taumax}{\tau_{\mathrm{max}}}
\newcommand{\ntil}{\widetilde{n}}
\newcommand{\Ntil}{\widetilde{N}}
\newcommand{\Ncheck}{\check{N}}
\newcommand{\kbar}{\overbar{k}}

\newcommand{\mmin}{m_{\mathrm{min}}}
\newcommand{\mmax}{m_{\mathrm{max}}}

% PartialSBM
\newcommand{\ellv}{\boldsymbol{\ell}}
\newcommand{\Cieq}{C_{1,\mathrm{eq}}}
\newcommand{\Cidif}{C_{1,\mathrm{dif}}}
\newcommand{\Ctilde}{\tilde{C}}
\newcommand{\dbar}{\overline{d}}
\newcommand{\bsigma}{\boldsymbol{\sigma}}
\newcommand{\Poisson}{\mathrm{Poisson}}
\newcommand{\ktilde}{\tilde{k}}
 
% GT_ISIT
\newcommand{\Veq}{V_{\mathrm{eq}}}
\newcommand{\Vdif}{V_{\mathrm{dif}}}
\newcommand{\Vveq}{\mathbf{V}_{\mathrm{eq}}}
\newcommand{\Vvdif}{\mathbf{V}_{\mathrm{dif}}}
\newcommand{\veq}{v_{\mathrm{eq}}}
\newcommand{\vdif}{v_{\mathrm{dif}}}

%
% MultiMRI
%
\newcommand{\bpsi}{\boldsymbol{\psi}}

%
% Group testing 
%
\newcommand{\Hg}{\mathrm{Hypergeometric}}
\newcommand{\TV}{\mathrm{TV}}
\newcommand{\dTV}{d_{\mathrm{TV}}}

%
% Graphical model selection
%
\newcommand{\Gbar}{\overbar{G}}
\newcommand{\Mbar}{\overbar{M}}
\newcommand{\qmax}{q_{\mathrm{max}}}

%
% Community detection
%
\newcommand{\Bi}{\mathrm{Binomial}}
\newcommand{\Hc}{\mathcal{H}}
\newcommand{\Ev}{\mathbf{E}}

%
% Support recovery
%
\newcommand{\Sbar}{\overbar{S}}
\newcommand{\sbar}{\overbar{s}}

%\newcommand{\Wdif}{W_{\mathrm{dif}}}
%\newcommand{\Weq}{W_{\mathrm{eq}}}
%\newcommand{\wdif}{w_{\mathrm{dif}}}
%\newcommand{\weq}{w_{\mathrm{eq}}}

\newcommand{\Sdif}{S_{\mathrm{dif}}}
\newcommand{\Seq}{S_{\mathrm{eq}}}
\newcommand{\sdif}{s_{\mathrm{dif}}}
\newcommand{\sdifbar}{\overbar{s}_{\mathrm{dif}}}
\newcommand{\seqbar}{\overbar{s}_{\mathrm{eq}}}
\newcommand{\seq}{s_{\mathrm{eq}}}
\newcommand{\sdifhat}{\hat{s}_{\mathrm{dif}}}
\newcommand{\sdifx}{\overbar{s} \backslash s}
\newcommand{\seqx}{\overbar{s} \cap s}

\newcommand{\Xvdif}{\mathbf{X}_{\overbar{s} \backslash s}}
\newcommand{\Xveq}{\mathbf{X}_{\overbar{s} \cap s}}
\newcommand{\Xvdifo}{\mathbf{X}_{s \backslash \overbar{s}}}
\newcommand{\xvdif}{\mathbf{x}_{\overbar{s} \backslash s}}
\newcommand{\xveq}{\mathbf{x}_{\overbar{s} \cap s}}
\newcommand{\xveqo}{\mathbf{x}_{s \backslash \overbar{s}}}

\newcommand{\Ord}{\mathcal{O}}
\newcommand{\Yctil}{\widetilde{\mathcal{Y}}}

\newcommand{\dmax}{d_{\mathrm{max}}}
\newcommand{\dH}{d_{\mathrm{H}}}

\newcommand{\jeq}{j_{\mathrm{eq}}}
\newcommand{\jdiff}{j_{\mathrm{dif}}}

\newcommand{\Zhat}{\hat{Z}}
\newcommand{\Zhatmax}{\hat{Z}_{\mathrm{max}}}
\newcommand{\zdiff}{z_{\mathrm{dif}}}
\newcommand{\zcomm}{z_{\mathrm{eq}}}

\newcommand{\bmin}{b_{\mathrm{min}}}
\newcommand{\Bmin}{\beta_{\mathrm{min}}}
\newcommand{\bmax}{b_{\mathrm{max}}}
\newcommand{\sigbeta}{\sigma_{\beta}}
\newcommand{\siglv}{\sigma_{\sdif}}
\newcommand{\Tr}{\mathrm{Tr}}

\newcommand{\sign}{\mathrm{sign}}
\newcommand{\SNRdB}{\mathrm{SNR}_{\mathrm{dB}}}

\newcommand{\Weq}{W_{\mathrm{eq}}}
\newcommand{\Wdif}{W_{\mathrm{dif}}}
\newcommand{\weq}{w_{\mathrm{eq}}}
\newcommand{\wdif}{w_{\mathrm{dif}}}
\newcommand{\feq}{f_{\mathrm{eq}}}
\newcommand{\fdif}{f_{\mathrm{dif}}}

\newcommand{\bgood}{b_{\mathrm{good}}}
\newcommand{\bbad}{b_{\mathrm{bad}}}

\newcommand{\btil}{\widetilde{b}}
\newcommand{\betatil}{\widetilde{\beta}}
\newcommand{\betatileq}{\widetilde{\beta}_{\mathrm{eq}}}
\newcommand{\betatildif}{\widetilde{\beta}_{\mathrm{dif}}}
\newcommand{\btileq}{\widetilde{b}_{\mathrm{eq}}}
\newcommand{\btildif}{\widetilde{b}_{\mathrm{dif}}}
\newcommand{\stileq}{\widetilde{s}_{\mathrm{eq}}}
\newcommand{\stildif}{\widetilde{s}_{\mathrm{dif}}}
\newcommand{\pieq}{\pi_{\mathrm{eq}}}
\newcommand{\pidif}{\pi_{\mathrm{dif}}}
\newcommand{\Sdifhat}{\hat{S}_{\mathrm{dif}}}

\newcommand{\Bcfano}{\mathcal{B}'_{\mathrm{Fano}}}

\newcommand{\Sceq}{\mathcal{S}_{\mathrm{eq}}}
\newcommand{\Scdif}{\mathcal{S}_{\mathrm{dif}}}
\newcommand{\Nc}{\mathcal{N}}

%
% From MultiMRI
%
\newcommand{\Pv}{\mathbf{P}}
\newcommand{\Wvtil}{\widetilde{\mathbf{W}}}
\newcommand{\wtil}{\widetilde{w}}
\newcommand{\xvhat}{\hat{\mathbf{x}}}

%
% From Expanders_FOCS
%
\newcommand{\Bernoulli}{\mathrm{Bernoulli}}
\newcommand{\Binomial}{\mathrm{Binomial}}
\newcommand{\imtil}{\tilde{\imath}}
\newcommand{\SNR}{\mathrm{SNR}}

%
% From PaperBI
%
\newcommand{\CLM}{C_{\mathrm{LM}}}
\newcommand{\CSC}{C_{\mathrm{SC}}}
\newcommand{\ISC}{I_{\mathrm{SC}}}
\newcommand{\CM}{C_{\mathrm{M}}}
\newcommand{\Qmin}{Q_{\mathrm{min}}}
\newcommand{\Phat}{\widehat{P}}

%
% From JournalSU
%
\newcommand{\pebar}{\overbar{P}_{\mathrm{e}}}
\newcommand{\petilde}{\tilde{p}_{\mathrm{e}}}
\newcommand{\pe}{P_{\mathrm{e}}}
\newcommand{\pen}{P_{\mathrm{e},n}}
\newcommand{\peQ}{P_{\mathrm{e},Q}} 
\newcommand{\pemax}{P_{\mathrm{e},\mathrm{max}}} 
\newcommand{\msg}{\mathsf{m}}

\newcommand{\Ez}{E_{0}}
\newcommand{\Er}{E_{\mathrm{r}}}
\newcommand{\Esp}{E_{\mathrm{sp}}}
\newcommand{\Eziid}{E_{0}^{\mathrm{iid}}}
\newcommand{\Ezcc}{E_{0}^{\mathrm{cc}}}
\newcommand{\Eohatcc}{\hat{E}_{0}^{\mathrm{cc}}}
\newcommand{\Eohatcost}{\hat{E}_{0}^{\mathrm{cost}}}
\newcommand{\Ezcost}{E_{0}^{\mathrm{cost}}}
\newcommand{\Ezcostprime}{E_{0}^{\mathrm{cost}^{\prime}}}
\newcommand{\Eriid}{E_{\mathrm{r}}^{\mathrm{iid}}}
\newcommand{\Ercc}{E_{\mathrm{r}}^{\mathrm{cc}}}
\newcommand{\Erccprime}{E_{\mathrm{r},12}^{\mathrm{cc}^{\prime}}}
\newcommand{\Ercost}{E_{\mathrm{r}}^{\mathrm{cost}}}
\newcommand{\Ercostprime}{E_{\mathrm{r}}^{\mathrm{cost}^{\prime}}}

\newcommand{\rcu}{\mathrm{rcu}}
\newcommand{\rcus}{\rcu_{s}}
\newcommand{\rcuss}{\rcu_{s}^{*}}
\newcommand{\rcusa}{\rcu_{s,a}^{*}}
\newcommand{\rcushat}{\widehat{\mathrm{rcu}}_{s}}
\newcommand{\rcusshat}{\widehat{\mathrm{rcu}}_{s}^{*}}
\newcommand{\alphasa}{\alpha^{\mathrm{iid}}}
\newcommand{\rcul}{\rcu_{\mathrm{L}}}
\newcommand{\rhohat}{\hat{\rho}}
\newcommand{\LM}{I_{\mathrm{LM}}}
\newcommand{\GMI}{I_{\mathrm{GMI}}}

\newcommand{\SetScc}{\mathcal{S}}
\newcommand{\SetSncc}{\mathcal{S}_{n}}
\newcommand{\SetSncost}{\mathcal{S}_{n}}
\newcommand{\SetSiid}{\mathcal{S}^{\mathrm{iid}}}
\newcommand{\SetScost}{\mathcal{S}}
\newcommand{\SetSniid}{\mathcal{S}_{n}^{\mathrm{iid}}}
\newcommand{\SetTcc}{\mathcal{T}}
\newcommand{\SetTiid}{\mathcal{T}^{\mathrm{iid}}}
\newcommand{\SetTcost}{\mathcal{T}}
\newcommand{\SetTncost}{\mathcal{T}_{n}}
\newcommand{\SetTncc}{\mathcal{T}_{n}}

\newcommand{\Miid}{M^{\mathrm{iid}}}
\newcommand{\Mcc}{M^{\mathrm{cc}}}
\newcommand{\Mcost}{M^{\mathrm{cost}}}
\newcommand{\Rcr}{R_{\mathrm{cr}}}
\newcommand{\Rcrs}{R_s^{\mathrm{cr}}}

\newcommand{\Fbar}{\overbar{F}}
\newcommand{\Funder}{\underline{F}}
\newcommand{\Gunder}{\underline{G}}
\newcommand{\PXv}{P_{\Xv}}
\newcommand{\qbar}{\overbar{q}}
\newcommand{\rbar}{\overbar{r}}
\newcommand{\rtilde}{\tilde{r}}
\newcommand{\atilde}{\tilde{a}}
\newcommand{\phitilde}{\tilde{\phi}}
\newcommand{\Mtilde}{\mytildee{M}}
\newcommand{\Ptilde}{\widetilde{P}}
\newcommand{\Qtilde}{\widetilde{Q}}
\newcommand{\Etilde}{\widetilde{E}}
\newcommand{\Rtilde}{\mytildee{R}}
\newcommand{\Wtilde}{\widetilde{W}}
\newcommand{\Wvtilde}{\widetilde{\mathbf{W}}}
\newcommand{\hover}{\overbar{h}}
\newcommand{\ubar}{\overbar{u}}
\newcommand{\Ubar}{\overbar{U}}
\newcommand{\xbarbar}{\overbar{\overbar{x}}}
\newcommand{\Xbarbar}{\overbar{\overbar{X}}}
\newcommand{\xbar}{\overbar{x}}
\newcommand{\Xbar}{\overbar{X}}
\newcommand{\xtilde}{\widetilde{x}}
\newcommand{\Xtilde}{\mytilde{X}}
\newcommand{\ybar}{\overbar{y}}
\newcommand{\zbar}{\overbar{z}}
\newcommand{\Zbar}{\overbar{Z}}
\newcommand{\thetabar}{\overbar{\theta}}

\newcommand{\av}{\mathbf{a}}
\newcommand{\Qv}{\mathbf{Q}}
\newcommand{\uvbar}{\overbar{\mathbf{u}}}
\newcommand{\uv}{\mathbf{u}}
\newcommand{\Uvbar}{\overbar{\mathbf{U}}}
\newcommand{\Uv}{\mathbf{U}}
\newcommand{\vv}{\mathbf{v}}
\newcommand{\Vv}{\mathbf{V}}
\newcommand{\vvbar}{\overbar{\mathbf{v}}}
\newcommand{\xvbarbar}{\overbar{\overbar{\mathbf{x}}}}
\newcommand{\xvbar}{\overbar{\mathbf{x}}}
\newcommand{\xvtilde}{\widetilde{\mathbf{x}}}
\newcommand{\xv}{\mathbf{x}}
\newcommand{\Xvbarbar}{\overbar{\overbar{\mathbf{X}}}}
\newcommand{\Xvbar}{\overbar{\mathbf{X}}}
\newcommand{\Xvtilde}{\mytilde{\mathbf{X}}}
\newcommand{\Ytilde}{\widetilde{Y}}
\newcommand{\Yvtilde}{\widetilde{\mathbf{Y}}}
\newcommand{\Xv}{\mathbf{X}}
\newcommand{\yv}{\mathbf{y}}
\newcommand{\Yv}{\mathbf{Y}}
\newcommand{\zv}{\mathbf{z}}
\newcommand{\Zv}{\mathbf{Z}}

\newcommand{\Yvs}{\mathbf{Y}_{\hspace*{-0.5ex}s}}
\newcommand{\Uvn}{\mathbf{U}_{\hspace*{-0.3ex}n}}
\newcommand{\Tvn}{\mathbf{T}_{\hspace*{-0.3ex}n}} 
\newcommand{\Vvi}{\mathbf{V}_{\hspace*{-0.3ex}1}} 
\newcommand{\Vvii}{\mathbf{V}_{\hspace*{-0.3ex}2}}
\newcommand{\Vvm}{\mathbf{V}_{\hspace*{-0.3ex}m}}
\newcommand{\Vvtilden}{\widetilde{\mathbf{V}}_{\hspace*{-0.3ex}n}}   
\newcommand{\QUn}{Q_{U\hspace*{-0.2ex},\hspace*{0.05ex}n}}
\newcommand{\QXgUn}{Q_{X|U\hspace*{-0.2ex},\hspace*{0.05ex}n}}
\newcommand{\QUXn}{Q_{UX\hspace*{-0.2ex},\hspace*{0.05ex}n}}
\newcommand{\Yvi}{\mathbf{Y}_{\hspace*{-0.5ex}1}}
\newcommand{\Yvii}{\mathbf{Y}_{\hspace*{-0.5ex}2}}
\newcommand{\Yvu}{\mathbf{Y}_{\hspace*{-0.5ex}u}}

\newcommand{\Ac}{\mathcal{A}}
\newcommand{\Bc}{\mathcal{B}}
\newcommand{\Cc}{\mathcal{C}}
\newcommand{\Dc}{\mathcal{D}}
\newcommand{\Ec}{\mathcal{E}}
\newcommand{\Fc}{\mathcal{F}}
\newcommand{\Gc}{\mathcal{G}}
\newcommand{\ic}{\mathcal{i}}
\newcommand{\Ic}{\mathcal{I}}
\newcommand{\Kc}{\mathcal{K}}
\newcommand{\Pc}{\mathcal{P}}
\newcommand{\Sc}{\mathcal{S}}
\newcommand{\Tc}{\mathcal{T}}
\newcommand{\Uc}{\mathcal{U}}
\newcommand{\Wc}{\mathcal{W}}
\newcommand{\Xc}{\mathcal{X}}
\newcommand{\Yc}{\mathcal{Y}}
\newcommand{\Zc}{\mathcal{Z}}

\newcommand{\CC}{\mathbb{C}}
\newcommand{\EE}{\mathbb{E}}
\newcommand{\PP}{\mathbb{P}}
\newcommand{\RR}{\mathbb{R}}
\newcommand{\ZZ}{\mathbb{Z}}
\newcommand{\Qsf}{\mathsf{Q}}

\newcommand{\al}{\{a_l\}}
\newcommand{\rl}{\{r_l\}}
\newcommand{\rbarl}{\{\overbar{r}_l\}}
\newcommand{\defeq}{\triangleq}
\newcommand{\deq}{\stackrel{d}{=}}
\newcommand{\var}{\mathrm{Var}}
\newcommand{\cov}{\mathrm{Cov}}
\newcommand{\erfcxi}{\mathrm{erfcx}_{1}}
\newcommand{\erfcx}{\mathrm{erfcx}}
\newcommand{\erfc}{\mathrm{erfc}}
\newcommand{\supp}{\mathrm{supp}}

\newcommand{\alphanl}{\alpha^{\mathrm{nl}}_n}
\newcommand{\alphal}{\alpha^{\mathrm{l}}_n}
\newcommand{\betanl}{\beta^{\mathrm{nl}}_n}
\newcommand{\betal}{\beta^{\mathrm{l}}_n}

\newcommand{\Es}{E_{\mathrm{s}}}
\newcommand{\EJ}{E_{\mathrm{J}}}
\newcommand{\EJi}{E_{\mathrm{J},1}}
\newcommand{\EJii}{E_{\mathrm{J},2}}

\newcommand{\Vmin}{V_{\mathrm{min}}}
\newcommand{\Vmax}{V_{\mathrm{max}}}

%
% From PaperExpurgated
%

\newcommand{\rcux}{\mathrm{rcux}}

\newcommand{\Eexiid}{E_{\mathrm{ex}}^{\mathrm{iid}}}
\newcommand{\Eexcc}{E_{\mathrm{ex}}^{\mathrm{cc}}}
\newcommand{\Eexhatcc}{\hat{E}_{\mathrm{ex}}^{\mathrm{cc}}}
\newcommand{\Eexcost}{E_{\mathrm{ex}}^{\mathrm{cost}}}
\newcommand{\Eexcostprime}{E_{\mathrm{ex}}^{\mathrm{cost}^{\prime}}}
\newcommand{\Exiid}{E_{\mathrm{x}}^{\mathrm{iid}}}
\newcommand{\Excc}{E_{\mathrm{x}}^{\mathrm{cc}}}
\newcommand{\Excost}{E_{\mathrm{x}}^{\mathrm{cost}}}
\newcommand{\Excoststar}{E_{\mathrm{x}}^{\mathrm{cost}^{*}}}
\newcommand{\Excostprime}{E_{\mathrm{x}}^{\mathrm{cost}^{\prime}}}
\newcommand{\Eex}{E_{\mathrm{ex}}}
\newcommand{\Ex}{E_{\mathrm{x}}}

\newcommand{\Pxbicm}{P_{X}^{\rm bicm}}

\newcommand{\Av}{\mathbf{A}}
\newcommand{\Bv}{\mathbf{B}}
\newcommand{\bv}{\mathbf{b}}
\newcommand{\mbar}{\overbar{m}}
\newcommand{\tv}{\mathbf{t}}

\newcommand{\Lsf}{\mathsf{L}}
\newcommand{\Csf}{\mathsf{C}}
\newcommand{\csf}{\mathsf{c}}
\newcommand{\Fsf}{\mathsf{F}}
\newcommand{\Tsf}{\mathsf{T}}

\newcommand{\brl}{\{\overbar{r}_l\}}

%
% From DocumentMU
%

\newcommand{\peobar}{\overbar{p}_{\mathrm{e},0}}
\newcommand{\peibar}{\overbar{p}_{\mathrm{e},1}}
\newcommand{\peiibar}{\overbar{p}_{\mathrm{e},2}}
\newcommand{\peiiibar}{\overbar{p}_{\mathrm{e},12}}
\newcommand{\peo}{P_{\mathrm{e},0}}
\newcommand{\pei}{P_{\mathrm{e},1}}
\newcommand{\peii}{P_{\mathrm{e},2}}
\newcommand{\peiii}{P_{\mathrm{e},12}}
\newcommand{\rcuo}{\mathrm{rcu}_{0}}
\newcommand{\rcui}{\mathrm{rcu}_{1}}
\newcommand{\rcuii}{\mathrm{rcu}_{2}}
\newcommand{\rcuiii}{\mathrm{rcu}_{12}}
\newcommand{\rcuiiiprime}{\mathrm{rcu}_{12}^{\prime}}
\newcommand{\rcuiiia}{\mathrm{rcu}_{12,1}}
\newcommand{\rcuiiib}{\mathrm{rcu}_{12,2}}

\newcommand{\Eiziid}{E_{0,1}^{\mathrm{iid}}}
\newcommand{\Eiiziid}{E_{0,2}^{\mathrm{iid}}}
\newcommand{\Eiiiziid}{E_{0,12}^{\mathrm{iid}}}
\newcommand{\Eiiiaziid}{E_{0,12,1}^{\mathrm{iid}}}
\newcommand{\Eiiibziid}{E_{0,12,2}^{\mathrm{iid}}}
\newcommand{\Eozcc}{E_{0,0}^{\mathrm{cc}}}
\newcommand{\Eizcc}{E_{0,1}^{\mathrm{cc}}}
\newcommand{\Eiizcc}{E_{0,2}^{\mathrm{cc}}}
\newcommand{\Eiiizcc}{E_{0,12}^{\mathrm{cc}}}
\newcommand{\Eiiizccprime}{E_{0,12}^{\mathrm{cc}^{\prime}}}
\newcommand{\Eiiizccpprime}{E_{0,12}^{\mathrm{cc}^{\prime\prime}}}
\newcommand{\Eiiiazcc}{E_{0,12,1}^{\mathrm{cc}}}
\newcommand{\Eiiibzcc}{E_{0,12,2}^{\mathrm{cc}}}
\newcommand{\Ehatiiiazcc}{\hat{E}_{0,12,1}^{\mathrm{cc}}}
\newcommand{\Eozcost}{E_{0,0}^{\mathrm{cost}}}
\newcommand{\Eizcost}{E_{0,1}^{\mathrm{cost}}}
\newcommand{\Eiizcost}{E_{0,2}^{\mathrm{cost}}}
\newcommand{\Eiiizcost}{E_{0,12}^{\mathrm{cost}}}
\newcommand{\Eiiiazcost}{E_{0,12,1}^{\mathrm{cost}}}
\newcommand{\Eiiibzcost}{E_{0,12,2}^{\mathrm{cost}}}
\newcommand{\Eiiizaccts}{E_{0,12,1}^{\text{cc-ts}}}

\newcommand{\Eiriid}{E_{\mathrm{r},1}^{\mathrm{iid}}}
\newcommand{\Eiiriid}{E_{\mathrm{r},2}^{\mathrm{iid}}}
\newcommand{\Eiiiriid}{E_{\mathrm{r},12}^{\mathrm{iid}}}
\newcommand{\Eiiiariid}{E_{\mathrm{r},12,1}^{\mathrm{iid}}}
\newcommand{\Eiiibriid}{E_{\mathrm{r},12,2}^{\mathrm{iid}}}
\newcommand{\Eorcc}{E_{\mathrm{r},0}^{\mathrm{cc}}}
\newcommand{\Eorccprime}{E_{\mathrm{r},0}^{\mathrm{cc}^{\prime}}}
\newcommand{\Eircc}{E_{\mathrm{r},1}^{\mathrm{cc}}}
\newcommand{\Eiircc}{E_{\mathrm{r},2}^{\mathrm{cc}}}
\newcommand{\Eiiircc}{E_{\mathrm{r},12}^{\mathrm{cc}}}
\newcommand{\Eiiiarcc}{E_{\mathrm{r},12,1}^{\mathrm{cc}}}
\newcommand{\Eiiibrcc}{E_{\mathrm{r},12,2}^{\mathrm{cc}}}
\newcommand{\Eorcost}{E_{\mathrm{r},0}^{\mathrm{cost}}}
\newcommand{\Eircost}{E_{\mathrm{r},1}^{\mathrm{cost}}}
\newcommand{\Eiircost}{E_{\mathrm{r},2}^{\mathrm{cost}}}
\newcommand{\Eiiircost}{E_{\mathrm{r},12}^{\mathrm{cost}}}
\newcommand{\Eiiiarcost}{E_{\mathrm{r},12,1}^{\mathrm{cost}}}
\newcommand{\Eiiibrcost}{E_{\mathrm{r},12,2}^{\mathrm{cost}}}
\newcommand{\Eiiirtilde}{\widetilde{E}_{\mathrm{r,12}}}
\newcommand{\Eiiirccts}{E_{\mathrm{r},12}^{\text{cc-ts}}}

\newcommand{\Wsu}{W^{\prime}}
\newcommand{\qsu}{q^{\prime}}
\newcommand{\RegLM}{\mathcal{R}_{\mathrm{LM}}}
\newcommand{\RegLMex}{\mathcal{R}_{\mathrm{ex}}}

\newcommand{\SetSccts}{\mathcal{S}^{\mathrm{ts}}}
\newcommand{\SetTocc}{\mathcal{T}_0}
\newcommand{\SetToncc}{\mathcal{T}_{0,n}}
\newcommand{\SetTicc}{\mathcal{T}_{1}}
\newcommand{\SetTiicc}{\mathcal{T}_{2}}
\newcommand{\SetTiiicc}{\mathcal{T}_{12}}
\newcommand{\SetTiiincc}{\mathcal{T}_{12,n}}
\newcommand{\SetTnucc}{\mathcal{T}_{\nu}}
\newcommand{\SetTnuccex}{\mathcal{T}_{\nu}^{\mathrm{ex}}}
\newcommand{\SetTiccex}{\mathcal{T}_{1}^{\mathrm{ex}}}
\newcommand{\SetTiiccex}{\mathcal{T}_{2}^{\mathrm{ex}}}
\newcommand{\SetTiiiccex}{\mathcal{T}_{12}^{\mathrm{ex}}}
\newcommand{\SetTiccts}{\mathcal{T}_{1}^{\mathrm{ts}}}
\newcommand{\SetTiiccts}{\mathcal{T}_{2}^{\mathrm{ts}}}
\newcommand{\SetTiiiccts}{\mathcal{T}_{12}^{\mathrm{ts}}}

\newcommand{\qv}{\mathbf{q}}
\newcommand{\Wv}{\mathbf{W}}
\newcommand{\Rc}{\mathcal{R}}
\newcommand{\Vc}{\mathcal{V}}

\newcommand{\Rsc}{R^{\mathrm{sc}}}
\newcommand{\Rmac}{R^{\mathrm{ex}}}
\newcommand{\Rosc}{R_{0}^{\mathrm{sc}}}
\newcommand{\Risc}{R_{1}^{\mathrm{sc}}}
\newcommand{\Rimac}{R_{1}^{\mathrm{ex}}}
\newcommand{\Riimac}{R_{2}^{\mathrm{ex}}}
\newcommand{\Usc}{U^{\mathrm{sc}}}
\newcommand{\Ucsc}{\mathcal{U}^{\mathrm{sc}}}
\newcommand{\Ximac}{X_{1}^{\mathrm{ex}}}
\newcommand{\Xcimac}{\mathcal{X}_{1}^{\mathrm{ex}}}
\newcommand{\Xiimac}{X_{2}^{\mathrm{ex}}}
\newcommand{\Xciimac}{\mathcal{X}_{2}^{\mathrm{ex}}}

\newcommand{\PrML}{\PP^{(\mathrm{ML})}}
\newcommand{\PrS}{\PP^{(\mathrm{S})}}
\newcommand{\PrGenie}{\PP^{(\mathrm{Genie})}}

%
% From MAC2OR
%
\newcommand{\Lc}{\mathcal{L}}
\newcommand{\cv}{\mathbf{c}}
\newcommand{\Dv}{\mathbf{D}}
\newcommand{\Iv}{\mathbf{I}}
\newcommand{\iv}{\mathbf{i}}
\newcommand{\Fv}{\mathbf{F}}
\newcommand{\fv}{\mathbf{f}}
\newcommand{\Mv}{\mathbf{M}}
\newcommand{\Jv}{\mathbf{J}}
\newcommand{\Kv}{\mathbf{K}}
\newcommand{\kv}{\mathbf{k}}
\newcommand{\Lv}{\mathbf{L}}
\newcommand{\Rv}{\mathbf{R}}
\newcommand{\sv}{\mathbf{s}}
\newcommand{\Sv}{\mathbf{S}}
\newcommand{\Tv}{\mathbf{T}}
\newcommand{\bzero}{\boldsymbol{0}}
\newcommand{\bone}{\boldsymbol{1}}
\newcommand{\bmu}{\boldsymbol{\mu}}
\newcommand{\bnu}{\boldsymbol{\nu}}
\newcommand{\blambda}{\boldsymbol{\lambda}}
\newcommand{\balpha}{\boldsymbol{\alpha}}
\newcommand{\bphi}{\boldsymbol{\phi}}
\newcommand{\bgamma}{\boldsymbol{\gamma}}
\newcommand{\bTheta}{\boldsymbol{\Theta}}
\newcommand{\bOmega}{\boldsymbol{\Omega}}
\newcommand{\bSigma}{\boldsymbol{\Sigma}}
\newcommand{\bLambda}{\boldsymbol{\Lambda}}
\newcommand{\bPsi}{\boldsymbol{\Psi}}
\newcommand{\bDelta}{\boldsymbol{\Delta}}
\newcommand{\bell}{\boldsymbol{\ell}}
\newcommand{\Qinv}{\mathsf{Q}_{\mathrm{inv}}}

%
% From RefinementCC
%
\newcommand{\Wocc}{\mathcal{W}_{0}^{\mathrm{cc}}}
\newcommand{\Wosymm}{\mathcal{W}_{0}^{\mathrm{symm}}}
\newcommand{\Wsymm}{\mathcal{W}^{\mathrm{symm}}}
\newcommand{\Qocc}{\mathcal{Q}_{0}^{\mathrm{cc}}}
\newcommand{\Qc}{\mathcal{Q}}
\newcommand{\bxi}{\mathbf{\xi}}
\newcommand{\Ezxcc}{E_{0,x}^{\mathrm{cc}}}
\newcommand{\Ezxicc}{E_{0,x_{i}}^{\mathrm{cc}}}
\newcommand{\Rcrsa}{R_{s,a}^{\mathrm{cr}}}

%
% For FYP
%
\newcommand{\RKHSGauss}{\mathcal{H}_\infty}
\newcommand{\half}{\frac{1}{2}}
\newcommand{\Trans}{\text{T}}
\setlength\parindent{0pt}

\title{MA4199 FYP -- Bias Variance Tradeoff}
\date{}
\author{Ng Wei Le}

%\newcommand{\ntrain}{n_{\rm train}}
%\newcommand{\ntest}{n_{\rm test}}
\newcommand\norm[1]{\left\lVert#1\right\rVert}
\newcommand\inner[1]{\langle#1\rangle}
%\DeclareMathOperator{\Tr}{Tr}
\onehalfspacing

\begin{document}
\maketitle
\section{Kernels} \label{sec:Kernels}
Notation: we use the symbol $\KK$  when it can refer to both $\RR$ or $\CC$. Also, let $z^*$ or $(z)^*$ denote the conjugate of $z$ for any $z \in \CC$.
The sections covering Kernels and reproducing kernel Hilbert spaces are heavily referenced using Steinwart, Christman \cite{steinwartSVM}.
\begin{defn}
	For a non-empty set $X$, let $k: X \times X \rightarrow \KK$ be known as a kernel if there exists a function $\phi: X \rightarrow \mathcal{H}$ (known as a feature map of k) where $\mathcal{H}$ is a $\KK$-Hilbert space (known as a feature space of k) such that
	\begin{equation}
	k(x_1, x_2) = \inner{\phi(x_2), \phi(x_1)}_\mathcal{H}.
	\end{equation}
\end{defn}

\begin{lem} \label{lem:kernelSymm}
	For any kernel $k$ on $X$, $k(x_1, x_2) = k(x_2, x_1)^*$.
\end{lem}

From the properties of the inner product, we know that $k(x_1, x_2) = \inner{\phi(x_1), \phi(x_2)}^* = k(x_2, x_1)^*$. Therefore, for kernels on $\RR$, the symmetric property: $k(x_1, x_2) = k(x_2, x_1)$ holds.
\begin{lem}
	Let $k_1, k_2$ be kernels on a non-empty set $X$. Then $k_1 + k_2$ and $ak_1, a \in \RR^+ \cup \{0\}$ are kernels. 
\end{lem}
Below, we define the Gaussian RBF kernel:
\begin{defn}
	Let the complex Gaussian RBF kernel (on $\CC^d$)be defined as:
	\[ k_{\gamma, \CC^d}(z, z') := e^{-\gamma^{-2} \sum_{i=1}^{d} (z_i - z_i'^*)^2 } .\]
	We then define the real Gaussian RBF kernel (or simply the Gaussian RBF kernel for short) acting on $\RR^d$ as:
	\[ k_\gamma(x, x') = e^{- \gamma^{-2} \norm{x - x'}_2^2} \]
\end{defn}
It can be shown (\cite{steinwartSVM}) that the complex and real Gaussian RBF kernels are kernels.
\begin{defn}
	For a non-empty set $X$, a function $k: X \times X \rightarrow \RR$ is said to be a positive definite if, for any $m \in \ZZ^+ \cup \{0\}$ and all $x_1, ..., x_n \in X$, we have the following matrix (called the Gram matrix) being positive semi-definite:
	\[ K := (k(x_i, x_j))_{i,j}. \]
	Equivalently: for all $a_1, ..., a_n \in \RR$, we have:
	\[ \sum_{j=1}^{n} \sum_{i=1}^{n} a_j a_i k(x_j, x_i). \]
\end{defn}
\begin{defn}
	The positive definite function $k: X \times X \rightarrow \RR$ is said to be symmetric if  $k(x_1, x_2) = k(x_2, x_1)$ for all $x_1, x_2 \in X$
\end{defn}
\begin{thm}
	A real function $k:X \times X \rightarrow \RR$ is a kernel if and only if $k$ is a positive definite symmetric function (also known as a positive definite kernel).
\end{thm}
\begin{proof}
	Suppose k is a kernel. Then there exists some feature map $\Phi: X \rightarrow \HH$.
	\begin{equation*}
	\begin{split}
		\sum_{j=1}^{n} \sum_{i=1}^{n} a_j a_i k(x_j, x_i) &= 
			\sum_{j=1}^{n} \sum_{i=1}^{n} a_j a_i \inner{\phi(x_i), \phi(x_j)}_\HH \\
			&= \inner{\sum_{i=1}^{n} a_i \phi(x_i), \sum_{j=1}^{n} a_j \phi(x_j)}_\HH \\
			&= \norm{\sum_{i=1}^{n} a_i \phi(x_i)} \\
			&\geq 0.
	\end{split}
	\end{equation*}
	Also, from Lemma \ref{lem:kernelSymm}, we know that the real kernel $k$ is symmetric, proving one side of the theorem. To prove the other side: \\
	Given $k:X \times X \rightarrow \RR$ a positive definite symmetric function, we prove that $\Phi: X \rightarrow H$ where $x \mapsto k(\cdot, x)$ is a valid feature map for some feature space $H$. First, we define
	\[ \hat{\HH} := \sum_{i=1}^{n} a_i k(\cdot, x_i), n \in \nonNegInt, a_i \in \RR \text{ for all } i, x_i \in X \text{ for all } i. \]
	For $f, g \in \hat{\HH}$ where $f = \sum_{i=1}^{n}a_i k(\cdot, x_i)$ and $g = \sum_{j=1}^{m}b_j k(\cdot, y_j)$, we define the inner product as such:\
	\begin{equation} \label{eqn:innerKernel}
	\begin{split}
		\inner{f, g}  &:= \sum_{i=1}^{n} \sum_{j=1}^{m} a_i b_j k(y_j, x_i) \\
			&= \sum_{j=1}^{m} b_j f(y_j) \\
			&= \sum_{i=1}^{n} a_ig(x_i) 
	\end{split}
	\end{equation}
	This definition is bilinear and symmetric. \\
	We also have: $\inner{f, f} = \sum_{i=1}^{n}\sum_{j=1}^{n}a_i a_j k(x_j, x_i) \geq 0 $ since k is a positive definite function. It can be shown that $\inner{\cdot, \cdot}$ follows Cauhy-Schwarz Inequality (\cite{steinwartSVM}), hence we have:
	\begin{equation*}
		\begin{split}
			|f(x)|^2 &= |\sum_{i=1}^{n}a_ik(\cdot, x_i)|^2 \\
				&= |\inner{f, k(\cdot, x)}|^2 ~ (\because (\ref{eqn:innerKernel}) \text{ with } g = \sum_{j=1}^{m}b_j k(\cdot, y_j) = k(\cdot, x) \text{ with } m=1, b_1 = 1, y_1 = x)\\
				&\leq \inner{k(\cdot, x), k(\cdot, x)} ~ \inner{f, f}. 
		\end{split}
	\end{equation*}
	Therefore, if $\inner{f, f} = 0$, then $f = 0$, hence showing that $\inner{f, f} > 0$ if and only if $f \neq 0$. Hence, $\inner{\cdot, \cdot}$ defines a proper inner product in $\hat{\HH}$. \\
	Let $\HH$ be the completion of $\hat{\HH}$ and the map $U: \hat{\HH} \rightarrow \HH$ be the map where $\inner{Ux, Uy}_{\HH} = \inner{x, y}_{\hat{\HH}} $ for all $x, y \in \hat{\HH}$. Then we have, for all $x, x' \in X$:
	\[ k(x, x') = \inner{k(\cdot, x'), k(\cdot, x)} _{\hat{\HH}} = \inner{Uk(\cdot, x'), Uk(\cdot, x)}_{\HH} \]. Thus we find a feature map of $k$, proving that $k$ is a kernel.
\end{proof}
\section{Reproducing Kernel Hilbert Spaces} \label{sec:RKHS}
Initially introduced by Stanislaw Zaremba, reproducing kernel Hilbert spaces have many applications in the fields such as Statistical Learning and complex analysis. An RKHS is a $\KK$-Hilbert function space where point evaluation is continuous linear funcitonal.
\begin{defn}
	\textbf{(RKHS).} Let $\mathcal{H}$ be a $\KK$-Hilbert space of functions over a non-empty set $X$. $\HH$ is called an RKHS over $X$ if the Dirac function $\delta_x: \HH \rightarrow \KK$ defined as:
	\[ \delta_x(f) := f(x), ~ x \in X, ~ f \in \HH \] is continuous.
	Equivalently, there exists $ 0 < M_x < \infty $ such that
	\[ \delta_x(f) \leq M_x \norm{f}_\mathcal{H}, ~ \text{for all} f \in \HH. \]
	$\delta_x$ is called a bounded operator on $\HH$.
\end{defn}

This is not easy to put into practice, hence the reproducing kernel is defined.
\begin{defn} \textbf{(Reproducing Kernel).}
	For a non-empty set $X$ and a function $k : X \times X \rightarrow \KK$ where $k(\cdot, x) \in \HH$ for all $x \in X$ and the following property hold for all $x in X$ and $f \in \HH$:
	\begin{equation} \label{eqn:ReproducingProp}
		f(x) = \inner{f, k(\cdot, x)}
	\end{equation}
	The condition in equation (\ref{eqn:ReproducingProp}) is also known as the reproducing property.
\end{defn}
\begin{defn} \label{def:CanFeatureMaps}
	\textbf{(Canonical Feature Maps).}
	Let $\mathcal{H}$ be an RKHS over $X$ with reproducing kernel $k$. Let the function $\Phi: X \rightarrow \mathcal{H}$ be defined such that for all $x \in X$,
	\[ \Phi(x) = k(\cdot, x). \]
	We call $\Phi$ the canonical feature map of $k$.
\end{defn}
\begin{lem}
	\textbf{(A reproducing kernel of an RKHS is a kernel).}
	Let $\mathcal{H}$ be an RKHS over $X$ with reproducing kernel $k$. Then $k$ is a kernel.
\end{lem}
\begin{proof}
	We simply proof that $\Phi$ is a feature map of $k$.
	\begin{equation*}
	\begin{split}
	\inner{\Phi(x_2), \Phi(x_1)} &= \inner{k(\cdot, x_2), k(\cdot, x_1)} \\
		&=  k(x_1, x_2) ~ (\because \text{Reproducing Property } (\ref{eqn:ReproducingProp}))
	\end{split}
	\end{equation*}
	So $\mathcal{H} is also a feature space of k$.
\end{proof}
\begin{lem}
	Let $\mathcal{H}$ be an $\KK$-Hilbert functional RKHS over $X$ with reproducing kernel $k$. Then $H$ is a Reproducing Kernel Hilbert Space.
\end{lem}
\begin{proof}
	Recall the Dirac functional $\delta_x: H \rightarrow \KK$ where:
	\[ \delta_x(f) = f(x), ~ x \in X, ~ f \in H. \]
	Then we have:
	\begin{equation*}
	\begin{split}
		|\delta_x(f)| &= |f(x)| \\
			&= |\inner{f, k(\cdot, x)}| ~ (\because \text{Reproducing Property }(\ref{eqn:ReproducingProp})) \\
			&\leq \norm{k(\cdot, x)}_\mathcal{H} \norm{f}_
			\mathcal{H} ~ (\because \text{Cauchy-Schwarz Inequality}) \\
	\end{split}
	\end{equation*}
	This shows that the Dirac functionals are continuous.
\end{proof}

\subsection{Representer Theorem} \label{subsec:RepThm}
Representer Theorem ensures that the $argmin$ of an empirical risk expression involving a function over an RKHS can be expressed as a linear combination of kernels applied on the training data points as proven in \cite{Representer_Theorem}.
\begin{thm} \label{thm:Representer}
	Given a non-empty set $X$, training data $\{(x_1,y_1), ... (x_n,y_n)\} \in X \times \RR$, and RKHS $\mathcal{H}$ be an $\RR$-Hilbert function space over $X$ with reproducing kernel $k:X \times X \rightarrow \RR$. Let $g$ be a strictly increasing function $g:[o, \infty]\rightarrow \RR$, and $l$ be an arbitrary loss function, where $l:(X \times \RR^2)^n \rightarrow \RR \cup \{\infty\}$. \\
	We want to minimize the following empirical risk term:
	\[ E(f, (x_1, y_1), ..., (x_n, y_n)) ~ := ~ l((x_1, y_1, f(x_1)), ..., (x_n, y_n, f(x_n))) + g(\norm{f}) .\]
	For $\hat{f} = argmin_{f \in \mathcal{H}} E(f, (x_1, y_1), ..., (x_n, y_n))$, $\hat{f}$ can be represented in the form:
	\[ \hat{f}(\cdot) = \sum_{i=1}^{n} a_i k(\cdot, x_i) \]
	with $a_i \in \RR$ for all $i$.
\end{thm}
\begin{proof}
	First we let $\Phi$ be the canonical feature map of $k$ as defined in \ref{def:CanFeatureMaps}. Recall: function $\Phi: X \rightarrow \mathcal{H}$ where $\Phi(x)(\cdot) = k(\cdot , x)$. Due to the reproducing property where $\Phi(x)(x') = \inner{\Phi(x), k(\cdot, x')}$, we have:
	\begin{equation*}
	\begin{split}
		\Phi(x)(x') &= k(x', x) \\
			&= \inner{\Phi(x), k(\cdot, x')} \\
			&= \inner{\Phi(x), \Phi(x')}. 
	\end{split}
	\end{equation*}
	So $\Phi$ is a feature space of $k$. Using orthogonal decomposition, we decompose $f \in \mathcal{H}$ into a component projected onto the span of ${\Phi(x_i), ..., \Phi(x_n)}$, and the other component orthogonal to this span. We will then prove this orthogonal component is $0$ for any $f$ that reduces the empirical risk  term, hence completing the prove.
	\[ f = \sum_{i=1}^{n} a_i \Phi(x_i) + \gamma, \]
	where $\gamma \in \mathcal{H}$, $\inner{\Phi(x_i), \gamma} = 0 \text{ for all } i$.\\
	Next, applying the reproducing property again,
	\begin{equation*}
	\begin{split}
	f(x_j) &= \inner{f, k(\cdot, x_j)} \\
	&= \inner{\sum_{i=1}^{n} a_i \Phi(x_i) + \gamma, \Phi(x_j)} \\
	&= \inner{\sum_{i=1}^{n} a_i \Phi(x_i),  \Phi(x_j)} + \inner{\gamma,  \Phi(x_j)} \\
	&= \sum_{i=1}^{n} a_i \inner{\Phi(x_i),  \Phi(x_j)}.
	\end{split}
	\end{equation*}
	Now, consider:
	\begin{equation*}
	\begin{split}
	\norm{f}^2 &= \norm{\sum_{i=1}^{n} a_i \Phi(x_i) + \gamma}^2 ~ (\because \text{orthogonality})\\
	&=  \norm{\sum_{i=1}^{n} a_i \Phi(x_i)}^2 + \norm{\gamma}^2 \\
	&\geq \norm{\sum_{i=1}^{n} a_i \Phi(x_i)}^2 \\
	\implies g(\norm{f}) &\geq g(\norm{\sum_{i=1}^{n} a_i \Phi(x_i)})
	\end{split}
	\end{equation*}
	Therefore, if we have $\gamma = 0$, since $f(x_i)$ is unaffected by this for all $i$, 
	$l((x_1, y_1, f(x_1)), ..., (x_n, y_n, f(x_n)))$ is also unaffected by $\gamma$. For the term $g(\norm{f})$, it decreases if we have $\gamma = 0$. Hence,  $\hat{f} = argmin_{f \in \mathcal{H}} E(f, (x_1, y_1), ..., (x_n, y_n))$, $\hat{f}$ must have $\gamma = 0$, and 
	\begin{equation*}
	\begin{split}
	\hat{f} &= \sum_{i=1}^{n} a_i \Phi(x_i) \\
		&=  \sum_{i=1}^{n} a_i k(\cdot, x_i)
	\end{split}
	\end{equation*}
\end{proof}


\section{Notations} \label{sec:Notations}

Let $\pi_m(\RR^d)$ be a multivariate polynomial with $d$ variables and degree $\leq m$, i.e. 
\[ \pi_m(\RR^d) = \{ p(x) = \sum_{k \leq m} c_k x^k \} \] 	
\\
Let $C^k(X)$ be the set of functions on $X$ that are $k$ times continuously differentiable. \\
For a point $x \in \RR^d$, it has the components of its coordinates $\chi_1, .., \chi_d$, whereas we represent $n$ points in $\RR^d$ as $x_1, .., x_n$.\\
We denote $\NN_0$ as the set of non-negative integers.
We denote the multi-index vector with its components as $\alpha = (\alpha_1, ..., \alpha_d)^\Trans \in \NN_{0}^{d}$, and $|\alpha| := \norm{\alpha}_1$ For $X \subseteq \RR^d$, $f \in C^k(X)$, $|\alpha| \leq k$ and $x \in \RR^d$, we denote:
\[ D^\alpha f := \frac{\partial^{|\alpha|}}{\partial\chi_1^{\alpha_1} \cdot\cdot\cdot \partial\chi_d^{\alpha_d}} f \]

We will define the power function, as defined in 11.2 in \cite{ScatteredDataApproximation}:
\begin{defn}
	Suppose $X \in \RR^d$ is open, with $k: X \times X \rightarrow \RR$ be a positive definite kernel. For $\alpha \in \NN_0^d$, $\hat{X} = {x_1, x_2, ..., x_n} \subseteq X$ the power function $P^{(\alpha)}_{k,\hat{X}}(x)$ is defined by:
	\begin{equation*}
	\begin{split}
		(P^{(\alpha)}_{k,\hat{X}}(x))^2 &:= D_1^\alpha D_2^\alpha k(x, x) 
			- 2 \sum_{j=1}^n D^{\alpha}u_j^*(x)D_1^\alpha k(x, x_j) \\
			&+ \sum_{i,j = 1}^{n} D^\alpha u_i^*(x) D^\alpha u_j^*(x) k(x_i, x_j).
	\end{split}
	\end{equation*}
\end{defn}

\begin{defn}
	The fill distance (or sometimes referred to as 'fill' for short) for a set of points $X = \{x_1, ..., x_N\} \subseteq \Omega$ for a bounded domain $\Omega$ is defined to be
	\[
	h_{X,\Omega} \coloneqq \sup_{x \in \Omega}\min_{1 \leq j \leq N} \norm{x- x_j}_2
	\].
\end{defn}

Theorem 11.22 in \cite{ScatteredDataApproximation}:
\begin{thm} \label{thm:interpolate}
Let $\Omega$ be a cube in $\RR^d$ and $k = \phi(\norm{\cdot}_2)$ be a positive definite kernel with $f = \phi(\cdot)$ satisfying the condition that there exists, $l_0$ and constant $M > 0$ such that for all $r > 0$ and $l > l_0$, $|f^{(l)}(r)| \leq l!M^l$. Then there exists a constant $c > 0$ such that the error between a function $f \in \RKHSGauss$ and its interpolant $s_{f,X}$ for all data points $X = \{x_1, ..., x_n\}$ can be bounded by:
\begin{equation*}
\norm{f - s_{f,X}}_{L_\infty(\Omega)} \leq \text{exp}(-c/h_{X,\Omega})|f|_{\RKHSGauss}
\end{equation*}
 with sufficiently small fill $h_{X,\Omega}$.
\end{thm}
\begin{proof}
	From the previous theorem, we have:
	\[ P^2(x) \leq [1 + c_1(2N)]^2 \norm{f - p}_{L_\infty(G)} \]
	Where $G$ is on the interval $[0, 4(c_2(2N))^2h^2]$, $x \in \Omega$, $p \in \pi_n(\RR)$, $h = h_{X, \Omega}$.
	
	From Theorem TODO: we have for sufficiently small fill distance $\fillDist \leq \frac{c_0}{2n}$, the constants $c_1, c_2$ can be replaced by:
	\begin{equation*}
	\begin{split}
	c_1(2n) &= \exp(2d\gamma_d(2n+1))\\
	c_2(2n) &= 2c_2n
	\end{split}
	\end{equation*}
	So $G$ is on the interval $[0, 16N^2c_2^2h^2]$
	For $p$ the Taylor series of $f$ about 0, and up to the term $t^N$, we then have:
	\begin{equation*}
	\begin{split}
		|f(t) - p(t)| &\leq t^{N+1}\frac{|f^{n+1}(t')|}{(n+1)!} \\
		&\leq M^{N+1}t^{N+1} ~ \text{ (By assumption)} \\
		\implies \norm{f - p}_{L_\infty(G)} &\leq (M \cdot 16N^2c_2^2h^2)^{N+1} \\
			&= (C_0 N^2 h^2)^{N+1} ~ \text{ for constant }C_0 = Mc_2^2\\
	\end{split}
	\end{equation*}
	Also, we have:
	\begin{equation*}
	\begin{split}
		[1 + c_1(2N)]^2 &= [1 + \exp(2d\gamma_d(2N + 1))]^2 \\
			&\leq [2\exp(2d\gamma_d(2N + 1))]^2 \\
			&= 4\exp(4d\gamma_d(2N + 1)) \\
			&= \exp(\log 4 + 4d\gamma_d(2N + 1)) \\
			&\leq \exp(C_1(N+1)) ~ \text{ for sufficiently large }C_1.\\
	\end{split}
	\end{equation*}
	We then have:
	\begin{equation} \label{eqn:P2bound}
	\begin{split}
	P^2_{\Phi, X}(x) &\leq [1 + c_1(2N)]^2 \norm{f - p}_{L_\infty(G)} \\
		&\leq \exp(C_1(N+1))( C_0 N^2 h^2)^{N+1} \\
		&= (C_0 N^2 h^2\exp(C_1))^{N+1} \\
		&= (C_2 N^2 h^2)^{N+1} ~ \text{ for constant }C_2 = C_0\exp(C_1). \\
	\end{split}
	\end{equation}
	For $C_3 = \min(\frac{c_0}{2}, \frac{1}{\sqrt{e C_2}})$, and $N$ such that
	\[ \frac{C_3}{N+1} \leq h \leq \frac{C_3}{N} \], which gives us:
	\begin{equation*}
	\begin{split}
		h &\leq \frac{c_0}{2N}, \\
		-(N+1) &\leq -C_3/h, \\
		N^2h^2 \leq &C_3^2 \leq \frac{1}{eC_2} \\
		\implies C_2N^2h^2 &\leq  1/e. \\
	\end{split}
	\end{equation*}
	We then have:
\begin{equation*}
		P^2_{\Phi, X}(x) \leq e^{-(N+1)} \leq e^{-C_3/h}. 
\end{equation*}
Now, using $C= C_3/2$ and $|f(x) - s_{f, X}(x)| \leq P_{\Phi, X}(x) |f|_{\RKHSGauss}$, we have:
\[ |f(x) - s_{f, X}(x)| \leq e^{-C/\fillDist} |f|_{\RKHSGauss}, \]
which is what we wanted to prove.
\end{proof}

\begin{thm}
	With the same conditions as Theorem \ref{thm:interpolate}, except that $f$ satisfies the stricter condition $|f^{(l)}(r)| \leq M^l$, we can get a better error bound of:
	\[ \norm{f - s_{f,X}}_{L_\infty(\Omega)} \leq \exp(\frac{c \log(\fillDist)}{\fillDist})\norm{f}_{\RKHSGauss}. \]
\end{thm}
\begin{proof}
	The inequality at \ref{eqn:P2bound} becomes:
	\[ P^2_{\Phi, X}(x) \leq  \frac{(C_2 N^2 h^2)^{N+1}}{(N+1)!}. \]

Using Stirling's inequality $1/n! \leq (e/n)^n$, we have:
\[ P^2_{\Phi, X}(x) \leq (eC_2 N h^2)^{N+1}. \]
Similarly, with $C_3 = \min(\frac{c_0}{2}, \frac{1}{e C_2})$, and $N$ such that
\[ \frac{C_3}{N+1} \leq h \leq \frac{C_3}{N}, \] we then get \[ eC_2Nh \leq 1, \] which gives us:
\[ P^2_{\Phi,X}(x) \leq h^{N+1} \leq h^{C_3/h} = e^{C_3 \log h / h}. \]
Following the steps of the previous theorem then gives us our result.
\end{proof}
\section{Approximation Theorem} \label{sec:AppThm}

The below theorem gives us some justification as to why the minimum norm interpolating function was chosen, though this only works under noiseless conditions:
% Might need to rephrase this part to avoid plag
\begin{thm} \label{thm:approx}
	Fix $h^* \in \RKHSGauss $  .
Let $(x_1,y_1), ..., (x_n,y_n)$ be i.i.d. random variables where $x_i$ drawn randomly from a compact cube $\Omega \subseteq \mathbb{R}^d $,
$y_i = h^*(x_i) \: \forall i$. There exists $A, B > 0$ such that for any interpolating $h \in \mathcal{H}_\infty $ with high probability
\begin{equation*}
\sup_{x \in \Omega} \vert h(x) - h^*(x)\vert < A e^{-B(n/log \, n)^{1/d}} (\| h^* \|_{\mathcal{H}_\infty} + \| h \|_{\mathcal{H}_\infty})
\end{equation*}
\end{thm}

With $h_{X,\Omega}$ as the fill on the order of $O(n/logn)^{-1/d}$ (using the theorem S1 in Belkin's paper which wasn't proved).
We consider $f(x) := h(x) - h*(x)$. Since $h$ is interpolating, we have $f(x_i) = 0$ for all $x_i$. We then let $s_{f,X}$ be the zero function, since it is an interpolant of $f$. Thus, we have:
$s_{f,X}$ can be bounded by:
\[
\norm{f}_{L_\infty(\Omega)} = \sup_{x \in \Omega} \vert h(x) - h^*(x)\vert <
\text{exp}(-c(n/log n)^{1/d})|f|_N(\Omega)
\]
\[
\leq \text{exp}(-c(n/log n)^{1/d}) (\| h^* \|_{\RKHSGauss} + \| h \|_{\RKHSGauss})
\]
Another form we can have is using proposition 14.1 in \cite{ScatteredDataApproximation}:
\begin{prop}
	Let $\Omega \subseteq \RR^d$ be bounded and measurable. Suppose $X = \{x_1, ... , x_N\} \subseteq \Omega$ is quasi-uniform with respect to  $c_{qu} > 0$. Then there exists constants $c_1, c_2 > 0$ depending only on space dimension $d$, on $\Omega$ and on $c_{qu}$ such that:
	\[c_1N^{-1/d} \leq h_{X,\Omega} \leq c_2N^{-1/d} \].
\end{prop}
With the definition of quasi-uniformness being:
\begin{defn}
	For the separation distance of $X = \{x_1, ... , x_N\}$ being defined as $q_x \coloneqq \half \min_{i\neq j} \norm{x_i - x_j}_2$.
\end{defn}
We can then use the above proposition with $n$ replacing $n/\text{log}n$.

In either case, by choosing a the smallest norm for $h$, we can see that it corresponds to the smallest upperbound for $\vert h(x) - h^*(x)\vert$.

\section{Existing Bounds Provide No Guarantees for Interpolated Kernel Classifiers} \label{sec:BoundsKernel}
Steps are: 
\begin{itemize}
	\item Find lower bound on function norm of t-overfitted classifiers in RKHS corresponding to Gaussian Kernels.
	\item Show loss for available bounds for kernel methods based on function norm (can perhaps use this to explain approximation theorem as well?)
\end{itemize}
Interpolation: 0 regression error. Overfitting: 0 classification error. Interpolation implies overfitting.
\begin{defn}
	We say $h \in H$ t-overfits data, if it achieves zero classification loss (overfits) and $\forall_iy_ih(x_i) > t > 0$.
\end{defn}
The below shows a theorem on how the function norm changes with respect to t-overfitting.
\begin{thm} \label{thm:normBound}
	Let $(\xv_i ,y_i)$ be data sampled from $P$ on $\Omega \times \{-1, 1\}$ for $i = 1,..., n$. Assume that $y$ is not a deterministic function of $x$ on a subset of non-zero measure. Then, with high probability, any $h$ that t-overfits the data, satisfies
	\[\norm{h}_H > Ae^{Bn^{1/d}}\]
	for some constants $A, B > 0$ depending on $t$.
\end{thm}

We define the $\gamma$-shattering and fat-shattering dimension below:
\begin{defn}
	Let F be a set of functions mapping from a domain $X$ to $\RR$. Suppose $S = \{x_1, x_2, ..., x_m\} \subseteq X$.  Suppose also that $\gamma$ is a positive real number. Then $S$ is $\gamma$-shattered by F if there are real
	numbers $r_1, r_2,..., r_m$, such that for each $b \in \{0, 1\}^m$ there is a function $f_b$ in $F$ with
	\[
	f_b(x_i) \geq r_i + \gamma \text{ if } b_i = 1 \text{, and } f_b(x_i) \leq r_i - \gamma \text{ if } b_i = 0 \text{, for } 1 \leq i \leq m.
	\] We say $r = (r_1, r_2,..., r_m)$ witnesses the shattering.
	Suppose that F is a set of functions from a domain $X$ to $\RR$ and that $\gamma > 0$. Then $F$ has $\gamma$-dimension $d$ if $d$ is the maximum	cardinality of a subset $S$ of $X$ that is $\gamma$-shattered by $F$. If no such maximum exists, we say that $F$ has infinite $\gamma$-dimension. The $\gamma$-dimension of $F$ is denoted $fat_F(\gamma)$. This defines a function $fat_F: \RR \rightarrow N \cup \{0,\infty\}$, which we call the fat-shattering dimension of $F$.
\end{defn}
\begin{proof}
Let $B_R = \{ f \in \mathcal{H}, \norm{f}_\mathcal{H} < R \}$ be a ball of radius $R$ in RKHS $\mathcal{H}$. Suppose the data is $\gamma$-overfitted, \cite{LossFATBound} gives us a high probability of a bound of
\[ L(f) < O(\frac{\text{ln}(n)^2}{\sqrt{n}}\sqrt{fat_{B_R}(\gamma/8)}) \] for $L(f)$ the expected classification error. Also, from \cite{ApproximationConcentration} we have \[fat_{B_R}(\gamma) < O((log(R/\gamma))^d)\].
We then have $B_R$ containing no function that $\gamma$ overfits the data unless
\[(log(R/\gamma))^d > O(n) \implies R > c_1 \text{ exp}(c_2(\frac{n}{\text{ln }n})^{1/d})\]
for some positive constants $c_1, c_2$.
\end{proof}
Classical bounds for kernel methods (\cite{UnderstandKernel} ) are in the form:
\[ \lvert \frac{1}{n} \sum_{i}l(f(x_i),y_i) - L(f) \rvert \leq C \frac{\norm{f}^a_\mathcal{H}}{n^b}, \hspace{1em} C,a,b \geq 0 \]
The right side on this will tend to infinity for bigger $\norm{f}_\mathcal{H}$, which is suggested by Theorem \ref{thm:normBound}.

\section{Random Fourier Features} \label{sec:RFFs}
For a feature map $\phi: \RR^d \rightarrow \RR^{d'}$ the kernel trick allows easy computation for positive definite kernel $k$ where $k(x,y) = <\phi(x), \phi(y)>$. We want to find a randomized feature map $z: \RR^d \rightarrow \RR^{\bar{d}}$ such that 
\[ k(x,y) = <\phi(x), \phi(y)> \approx <z^{\Trans}(x), z(y)> \].
As suggested by \cite{RFF_Rahimi}, for a shift-invariant kernel $k$: $k(x, y) = k(x - y)$), we consider the mapping $z(x) = cos(w^{\Trans}x + b)$, where $w$ is drawn from the probability distribution $p$:
\begin{equation}
p(w) = \frac{1}{2\pi} \int k(h) ~ \text{exp}(-iw^{\Trans}h) ~ \text{d}h
\label{eq:probFourier}
\end{equation} 
 
when we compute the Fourier transform of the kernel $k$, and $b$ is drawn from the uniform distribution on $[0, 2\pi]$.

We know that the fourier transform of $k(\cdot)$ is a probability distribution from Bochner's theorem:
\begin{thm} \label{thm:Bochner}
	(Bochner \cite{Rudin_1990}).For a continuous kernel $k(x - y)$  it is a positive definite kernel if and only if $k(\cdot)$ is the fourier transform of a non-negative measure.
\end{thm}
We now have:
\[ k(x - y) = \int_{\RR^d} p(w) \text{exp}(iw^{\Trans}(x - y)) ~ \text{d}w = \EE_w[e^{iw^{\Trans}x}(e^{iw^{\Trans}y})^*] \].
Therefore, we can use $e^{iw^{\Trans}x}(e^{iw^{\Trans}y})^*$ as an estimate (unbiased) of $k(x, y)$.
Let $\phi_w(x) = e^{iw^{\Trans}x}$
We can also use $z_w(x) = \sqrt{2}cos(w^{\Trans}x + b)$ instead of $\phi_w(x)$, as suggested by \cite{RFF_Rahimi}.
\begin{prop}
	For $z_w(x) = \sqrt{2}cos(w^{\Trans}x + b)$, where $w$ is drawn from probability distribution $p$ in \eqref{eq:probFourier} and $b$ drawn from a uniform random variable on $[0, 2\pi]$.
	\[E(z_w(x))z_w(y) = k(x,y)\]
\end{prop}
\begin{proof}
	\begin{equation*}
	\begin{split}
		z_w(x) &= 2 ~ \frac{\sqrt{2}}{2}cos(w^{\Trans}x + b) \\
		&= \frac{1}{\sqrt{2}}~(e^{i(w^{\Trans}x+b)} + e^{-i(w^{\Trans}x+b)}) \\
		&= \frac{1}{\sqrt{2}}~(\phi_w(x)e^{ib} + \phi_w(x)^* e^{-ib})
	\end{split}
	\end{equation*}
	Where $\phi_w(x) = e^{iw^{\Trans}x}$.
	\begin{equation*}
	\begin{split}
	z_w(x)z_y(y) &= \frac{1}{2}[\phi_w(x)\phi_w(y)e^{i2b} + \phi_w(x)^*\phi_w(y)^*e^{-i2b}
	 + \phi_w(x)\phi_w(y)^* + \phi_w(x)^*\phi_w(y)] \\
	\EE[z_w(x)z_y(y)] &= \frac{1}{2}\EE[\phi_w(x)\phi_w(y)e^{i2b} + \phi_w(x)^*\phi_w(y)^*e^{-i2b}]
	+ \frac{1}{2}\EE[\phi_w(x)\phi_w(y)^*] + \frac{1}{2}\EE[\phi_w(x)^*\phi_w(y)] \\ 
	\end{split}
	\end{equation*}
	As mentioned earlier in Theorem \ref{thm:Bochner}, $\EE_w[\phi_w(x)\phi_w(y)^*] = k(x - y)$.
	Also $\phi_w(x)\phi_w(y)^* = (\phi_w(x)^*\phi_w(y))^*$.
	\begin{equation*}
	\begin{split}
	\EE[z_w(x)z_y(y)] &= \frac{1}{2}\EE[\phi_w(x)\phi_w(y)e^{i2b} + \phi_w(x)^*\phi_w(y)^*e^{-i2b}]
	+ \frac{1}{2}k(x-y) + \frac{1}{2}[k(x-y)]^* \\
	&= \frac{1}{2}\EE[\phi_w(x)\phi_w(y)e^{i2b} + \phi_w(x)^*\phi_w(y)^*e^{-i2b}] + k(x-y)
	\end{split}
	\end{equation*}
	For real kernel , $k(x - y) = (k(x - y))^*$.
	\begin{equation*}
	\begin{split}
	\EE_{w,b}[\phi_w(x)\phi_w(y)e^{i2b}] &= \frac{1}{2\pi}\int_{\RR^d}\int_{0}^{2\pi} p(w)\phi_w(x)\phi_w(y)e^{i2b} \text{d}b~\text{d}w \\
	&= \frac{1}{2\pi}\int_{\RR^d} p(w)\phi_w(x)\phi_w(y) \int_{0}^{2\pi} e^{i2b} \text{d}b~\text{d}w \\
	&= 0
	\end{split}
	\end{equation*}
	Since $\int_{0}^{2\pi} e^{i2b} \text{d}b = 0$. Similarly, $	\EE_{w,b}[\phi_w(x)^*\phi_w(y)^*e^{-i2b}] = 0$.
	\[ \therefore \EE[z_w(x)z_y(y)] = k(x - y). \]
	
\end{proof}
As suggested by \cite{RFF_Rahimi}, the variance of the estimate is decreased by using $z$, a $D$ dimensional vector by concatenating $D$ of $z_w$ and normalizing by a constant $\sqrt{D}$. We let:
\[z(x) = \sqrt{\frac{2}{D}} [cos(w_1^{\Trans}x + b_1) ... cos(w_D^{\Trans}x + b_D)] \]
with randomly drawn $w_i$ and $b_i$ as described previously.


\begin{thm} \label{thm:RFFnorm}
	For $N$ the number of random features, and $x_1, x_2, ..., x_n$ the data points, when $N > n$ and as $N$ increases, the norm of the minimizer tends to  the norm of the minimum norm RKHS interpolant.
\end{thm}
\begin{proof}
Let $f(x)$ be the minimum norm RKHS interpolant function for the datapoints. 
\[f(x) = \sum_{i}\alpha_ik(x_i, x) \approx \sum_{i}\alpha_iz(x_i)^\text{T}z(x) 
 = \beta^\text{T}z(x) = \hat{f}(x)
\] (the first equality holds due to Representer Theorem)
Where $\beta = \sum_{i}\alpha_iz(x_i) $.
The norm of the function from the random fourier features approximation is:
\[ \norm{\beta} =   \beta^\text{T} \bar{\beta} = 
(\sum_{i}\alpha_i z^\text{T}(x_i)) (\sum_{i}\bar{\alpha}_i\bar{z}(x_i)) 
= \sum_i \sum_j \alpha_i \bar{\alpha}_j z^\text{T}(x_i)\bar{z}(x_j)  
\approx \sum_i \sum_j \alpha_i \bar{ \alpha_j} k(x_i, x_j)
= \norm{f}
\]
\end{proof}
\newpage
\bibliographystyle{plain}
\bibliography{refs}

\newpage
{\huge \centering \bf Appendix \par}

\appendix

\end{document} 


